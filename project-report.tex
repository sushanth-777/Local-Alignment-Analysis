\documentclass[a4paper,12pt]{article}
\usepackage[utf8]{inputenc}
\usepackage{amsmath}
\usepackage{graphicx}
\usepackage{hyperref}
\usepackage{geometry}
\geometry{margin=1in}

% Title and author information
\title{Comparative Analysis of Local Sequence Alignment Algorithms: BLAST, FASTA, and Smith-Waterman}
\author{Sushanth Reddy Kotha, Durga Sai Surya Ram Saladi, Mayur Sai Yaram}
\date{\today}

\begin{document}

\maketitle

\begin{abstract}
This report provides a comparative analysis of three sequence alignment algorithms—BLAST, FASTA, and Smith-Waterman—based on criteria such as speed, sensitivity, accuracy, and use cases. The goal is to understand each algorithm’s suitability for specific types of local sequence alignments in bioinformatics.
\end{abstract}

\section{Introduction}
Sequence alignment plays a crucial role in bioinformatics, helping identify similar regions between DNA or protein sequences. This report compares BLAST, FASTA, and Smith-Waterman to evaluate their performance on various criteria relevant to local sequence alignment.

\section{Methods}
\subsection{Dataset}
Describe the dataset used, including the types and sources of DNA or protein sequences.

\subsection{Tools}
Explain each alignment tool:
\begin{itemize}
    \item \textbf{BLAST}: Overview and basic functionality.
    \item \textbf{FASTA}: Description and working principles.
    \item \textbf{Smith-Waterman}: Details of the exact pairwise alignment algorithm.
\end{itemize}

\subsection{Alignment Criteria}
List the criteria used for comparison:
\begin{itemize}
    \item Algorithm type
    \item Speed and efficiency
    \item Alignment sensitivity
    \item Accuracy
    \item Use cases
\end{itemize}

\section{Results}
Present your findings for each tool. Use tables or figures as needed.

\section{Results}
Present your findings for each tool. Use tables or figures as needed.

\subsection{Score}
Table \ref{table:score} shows the score values for each sequence across the three alignment tools: BLAST, FASTA, and Smith-Waterman.

\begin{table}[h!]
\centering
\begin{tabular}{|l|c|c|c|}
    \hline
    \textbf{Sequence} & \textbf{BLAST Score} & \textbf{FASTA Score} & \textbf{Smith-Waterman Score} \\
    \hline
    
    BAAFST010000044.1 & 1729 & 222.1 & 90 \\
    Sequence 2 & 78 & 75 & 88 \\
    Sequence 3 & 92 & 89 & 95 \\
    Sequence 4 & 80 & 78 & 85 \\
    Sequence 5 & 87 & 84 & 91 \\
    \hline
\end{tabular}
\caption{Score values for each alignment tool.}
\label{table:score}
\end{table}

\subsection{Percentage Identity}
Table \ref{table:identity} shows the percentage identity for each sequence across BLAST, FASTA, and Smith-Waterman.

\begin{table}[h!]
\centering
\begin{tabular}{|l|c|c|c|}
    \hline
    \textbf{Sequence} & \textbf{BLAST Identity (\%)} & \textbf{FASTA Identity (\%)} & \textbf{Smith-Waterman Identity (\%)} \\
    \hline
    BAAFST010000044.1 & 78.28\% & 41.5\% & 97\% \\
    Sequence 2 & 89\% & 88\% & 90\% \\
    Sequence 3 & 92\% & 90\% & 95\% \\
    Sequence 4 & 85\% & 83\% & 89\% \\
    Sequence 5 & 91\% & 90\% & 93\% \\
    \hline
\end{tabular}
\caption{Percentage identity for each alignment tool.}
\label{table:identity}
\end{table}

\subsection{E-value}
Table \ref{table:evalue} presents the E-value for each sequence alignment across the three tools.

\begin{table}[h!]
\centering
\begin{tabular}{|l|c|c|c|}
    \hline
    \textbf{Sequence} & \textbf{BLAST E-value} & \textbf{FASTA E-value} & \textbf{Smith-Waterman E-value} \\
    \hline
    BAAFST010000044.1 & 0.0 & 6.1e-55 & 0.0001 \\
    Sequence 2 & 0.005 & 0.008 & 0.002 \\
    Sequence 3 & 0.0005 & 0.002 & 0.0002 \\
    Sequence 4 & 0.007 & 0.010 & 0.005 \\
    Sequence 5 & 0.002 & 0.006 & 0.001 \\
    \hline
\end{tabular}
\caption{E-value for each alignment tool.}
\label{table:evalue}
\end{table}

\subsection{Length of Alignment}
Table \ref{table:length} shows the length of alignment for each sequence across the three tools.

\begin{table}[h!]
\centering
\begin{tabular}{|l|c|c|c|}
    \hline
    \textbf{Sequence} & \textbf{BLAST Length} & \textbf{FASTA Length} & \textbf{Smith-Waterman Length} \\
    \hline
    BAAFST010000044.1 & 4963 & 526 & 152 \\
    Sequence 2 & 140 & 138 & 143 \\
    Sequence 3 & 160 & 157 & 162 \\
    Sequence 4 & 130 & 128 & 135 \\
    Sequence 5 & 155 & 153 & 158 \\
    \hline
\end{tabular}
\caption{Length of alignment for each alignment tool.}
\label{table:length}
\end{table}

\subsection{Execution Time}
Table \ref{table:time} presents the execution time for each sequence alignment across the three tools.

\begin{table}[h!]
\centering
\begin{tabular}{|l|c|c|c|}
    \hline
    \textbf{Sequence} & \textbf{BLAST Time (s)} & \textbf{FASTA Time (s)} & \textbf{Smith-Waterman Time (s)} \\
    \hline
    BAAFST010000044.1 & 0.5 & 0.7 & 1.2 \\
    Sequence 2 & 0.4 & 0.6 & 1.1 \\
    Sequence 3 & 0.6 & 0.8 & 1.3 \\
    Sequence 4 & 0.5 & 0.7 & 1.0 \\
    Sequence 5 & 0.7 & 0.9 & 1.4 \\
    \hline
\end{tabular}
\caption{Execution time for each alignment tool.}
\label{table:time}
\end{table}

\subsection{BLAST Analysis}
Summarize results obtained from BLAST, including scores, E-values, identities, and positives.

\subsection{FASTA Analysis}
Summarize the findings from FASTA and compare them with BLAST results.

\subsection{Smith-Waterman Analysis}
Describe the results from Smith-Waterman pairwise alignments, focusing on sensitivity and accuracy.

\section{Discussion}
Compare the three methods based on the specified criteria. Discuss the strengths and limitations of each, highlighting when each tool is most appropriate.

\section{Conclusion}
Summarize the findings and recommend the appropriate use cases for each algorithm.

\section{References}
\bibliographystyle{plain}
\begin{thebibliography}{9}
    \bibitem{blast}
    S. F. Altschul et al., "Basic local alignment search tool," Journal of Molecular Biology, vol. 215, no. 3, pp. 403–410, 1990.

    \bibitem{fasta}
    W. R. Pearson and D. J. Lipman, "Improved tools for biological sequence comparison," Proceedings of the National Academy of Sciences, vol. 85, no. 8, pp. 2444–2448, 1988.

    \bibitem{smith_waterman}
    T. F. Smith and M. S. Waterman, "Identification of common molecular subsequences," Journal of Molecular Biology, vol. 147, no. 1, pp. 195–197, 1981.
\end{thebibliography}

\end{document}
